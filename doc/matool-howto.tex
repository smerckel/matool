\documentclass{article}

\author{Lucas Merckelbach\\lucas.merckelbach@hzg.de}
\title{Quick guide to matool\\{\small Version 0.31}}

\newcommand{\mt}{\textsc{Matool}}
\newcommand{\mtc}{\textsc{matool}}
\newcommand{\mts}{\textsc{ma\_edit\_server}}
\begin{document}
\maketitle
\tableofcontents
\section{Introduction}
\mt{} is a simple tool aimed at facilitating the preparation,
uploading and versioning of glider ma and mi files. Its use allows
that if a glider is piloted by several people, they can get an
overview of the status of a glider (what files are active, or being
set for uploading), comments and issues for that specific glider. In
addition the \mt{} provides an interface for external programs to
query the contents of the activated mission and ma-files.

\subsection{Structure}
The structure of the \mt{} is a server-client model. One instance of
the server is running on the same machine where the dockserver is
running. The server has access to the \emph{to-glider} directories and
maintains a repository for each glider. The client, of which many
instances may be running on different computers, contacts the server
with requests, such as show the contents of a given ma-file for a
given glider, or open such a file for editing. The server will do what
is necessary. For example, if a request for editing is received, the
server will return the contents of the file to be edited. If the
client returns a modified version, the server will put the mafile in
the correct directory for uploading, and add the file to the
repository. Once the glider picked up the file, the server will notice
this and mark the file as active.

As always with communications between server and clients, a firewall
may complicate matters in that a firewall may prevent communications
between server and client. A common trick is circumvent such problems
is the use of tunnels. In this case, the program ssh is used to setup
securely tunnels that forward the server ports to the local
machine. This is simple and incorporated in \mt{} when running under
linux. On windows machines, things are more complicated, and a VPN may
provide the solution, or using the windows client putty to set up the
tunnels manually.

\section{Installation}
\subsection{Linux}
\mt{} is written in python, using only standard modules (no further
dependencies). Installation is straight-forward and conform the
standard for installing python modules.

\begin{enumerate}
\item Unpack the tgz file: tar xvzf matool-xxx.tar.gz
\item Cd into the newly created directory
\item run python setup.py build
\item run as root python setup.py install
\end{enumerate}
This requires root privileges, and installs \mt{} systemwide. \mt{}
can be installed for the current user only. Then, the option
``--prefix=/some/non-standard/directory/ needs to be appended to step
3, and step 4 can be run as normal user.

The installation process will install both the client and the server,
as well some common modules.
\subsection{Windows}
To be done.

\section{Matool, the client}
The \mt{} client is called \mtc{}. Currently it has only a text
interface. It can be run with a command, which is then exectuted and
the program exits, or it can be run without a command. When run
without a command, a menu and a prompt are provided, allowing the user
to formulate one or more commands. At this stage the user interface is
fairly basic and lacks features as auto-completion and history.

\subsection{\mtc{} run without commands}
If \mtc{} is run without a command, the following menu is displayed:
\begin{verbatim}
edit [rev <revision_number>] <mafile> <glider> [author]
show <current|queue> <mafile> <glider>
show rev <revision_number> <mafile> <glider>
show log <glider>
queue [glider]
list <glider>
delete <mafile> <glider>
log <glider> [author]
diff <mafile> <glider> <rev_x> [rev_y]
import [[[filename] [filename] ...] <glider>
export <current|queue> <mafile> <glider>

ls
help [command]

m | menu to display this menu
q | quit to quit
\end{verbatim}
This list shows all available commands such as edit, show etc. and the
arguments that each command can take. The convenction used is that
each argument between \verb+< >+ is required, and that arguments
between \verb+[ ]+ are optional. Order in which arguments are supplied
matters.

\subsubsection*{edit [rev $<$revision\_number$>$] $<$mafile$>$ $<$glider$>$
  [author]}
The edit command is used to modify an mafile. The required arguments
are the name of the mafile and the name of the glider. Optionally a
revision number can be supplied. In that case, the mafile with the
corresponding revision number will be used as basis, instead of the
current one. 

The author name is optional. If none is provided the login name of the
user is used. The choice of editor can be made user dependent, see
also ?????

If this command is run, the user is presented with an editor and the
contents of the mafile to be modified. If all modifications are
finished, the file is saved and the editor closed. The program will
ask the user to propagate the changes. If answered yes, then the
changes return to the server, which creates a file in the to-glider
directory, and adds the file to its repository (with a new version
number). The file is now ready for being picked up by the glider. Note
that the user is responsible for instructing the dockserver to upload
the mafile to the glider (by running the appropriate xml-script).

\subsubsection*{show $<$current|queue$>$ $<$mafile$>$ $<$glider$>$}
Shows the requested ma-file for given glider. Depending on the
argument current or queue, the contents of the last uploaded file
(current), or the contents of the file in the to-glider directory (queue) is
returned.

\subsubsection*{show rev $<$revision\_number$>$ $<$mafile$>$ $<$glider$>$}
Shows the requested ma-file for given glider for given revision
number. The file is retrieved from the repository.

\subsubsection*{show log $<$glider$>$}
Returns the log entries for specified glider. See also the log command.
\subsubsection*{queue [glider]}
Returns the filenames of all files in the to-glider directory. If
glider is not specified, all glider directories are searched.
\subsubsection*{list $<$glider$>$}
Lists all files currently known in the repository. If a file is marked
with a *, then it is not known what the contents is of the corresponding
file on the glider. I.e., the file has not been uploaded yet.

\subsubsection*{delete $<$mafile$>$ $<$glider$>$}
Removes $<$filename$>$ from the 'to-glider' directory of glider $<$glider$>$.
This effectively cancels the 'edit' action.

\subsubsection*{log $<$glider$>$ [author]}
Opens an editor with the log file for given glider. This log file can
be edited with remarks and comments. Upon saving and quitting the
editor, the log file on the server is updated.
\subsubsection*{diff $<$mafile$>$ $<$glider$>$ $<$rev\_x$>$ [rev\_y]}
Shows the difference between two revisions or one revision and current
file.

\subsubsection*{import [[[filename] [filename] ...] $<$glider$>$}
Imports given filenames for $<$glider$>$. If no filename(s) are given, 
all filenames ending with ma or mi found in the current directory are
imported. Note that this imports any ma files in the current directory
on the computer from which the client is running.

\subsubsection*{export $<$current|queue$>$ $<$mafile$>$ $<$glider$>$}
current:
  Saves the contents of the latest file that has been uploaded to the 
  glider, i.e. the contents of the file that should be active on the
  glider, to the local directory.

\noindent queue:
  Saves the contents of the file that is queued for uploading to the glider
  to the local directory.

\subsubsection*{ls}
This command is a for convenience; it lists all files in the current
direction with extension .ma or .mi.

\subsubsection*{help [command]}
The help command is the built-in help system. Without any commands, it
displays general help. If a valid command is supplied, the detailed
help of that command is displayed.

\subsubsection*{menu}
The menu command (or the short-hand m) gives a list of all commands
and their invocation. This command can be requested any time.

\subsubsection*{quit}
The quit command has the expected result. A short-hand is just the q.

\subsection{\mtc{} run with commands}
If \mtc{} is run with a command, no menu is displayed, but the command
is executed immediately, and upon completion, the command line is
returned. Only the commands ``edit'', ``show'', ``queue'', ``list'',
``delete'', ``log'', ``diff'', ``import'' and ``export'' are vaild
commands.

\subsection{Firewall issues}
Normally, the client would connect directly to port to which the
\mts{} listens. When invoked from the same machine, or within the
intranet, this should work as intended. If, on the other hand, the
client is run outside the intranet, a firewall has to be
by-passed. For linux, the most appropriate way to do this is to use
ssh to setup tunnels. On Windows a similar mechanism can be done,
using the ssh client putty. For linux, \mtc{} can  be configured such
that the tunnels are setup automatically.

The basic idea behind tunnels is that the ssh-client acts as a program
that listens to a port on the local machine (that is, the machine
where the ssh-client and thus \mtc{} is run). If a program connects to
this port, the ssh client forwards this port to a machine on the
remote end. This can be the machine which ssh logs in to, but also a
machine that is only visible by the remote machine (as it is behind
the firewall) and not from the local machine. For this method to work,
the user should have a valid user account on the server
bastion.hzg.de. Without such an account, no connection can be made
from outside the hzg network.

\subsection{Configuration file}
The \mtc{} can be configured using a configuration file .matoolrc,
stored in the user's home directory. If it doesn't exist on startup, a
configuration file is created using a default setting.

A typical configuration file looks like:
\begin{verbatim}
[Network]
host = 141.4.0.159
port = 9000

[Ssh]
sshserver = bastion.hzg.de
username = merckelb
gracetime = 300

[lucas]
editor = /usr/bin/gedit
\end{verbatim}

The configuration file consists of a section Network (mandatory) and
an optional section Ssh. The section Network has two parameters, host,
and port. The host has as argument the name or ip address of the
machine that is running the \mts{}. If \mtc{} is taking care
of the ssh tunneling, then this must be the real address of that
machine, even if it is not visible or connectable from the local
machine. The port number refers to the port number the \mts{} is
listening too, and must be 9000. 

The Ssh section is optional, but usually required if \mtc{} is run
outside the hzg network. The sshserver argument takes the name or ip
address of the ssh server to which the connection from outside the hzg
network is made (bastion.hzg.de). If the username on that server is
different from the user name on the local machine, then this can be
specified by the argument username. The argument gracetime sets the
time in seconds how long a connection should be kept alive. (If a
connection times out, a new attempt will be made.)

Then any number of sections can be defined where the name of the
section corresponds to a username. Only one argument can be specified:
the path to the preferred editor. The default editor is gedit, but if
any other editor is preferred (emacs, vi, nano, etc.) then this can be
specified on a per user basis.

If \mtc{} is not supposed to take care of any tunneling, then,
sshserver in section Ssh should be ``none'' or, the whole section Ssh
should not exist. In such case, host and port should take the values
that match the tunnel definition used. 

An example of such a situation is that when putty is used to setup the
tunnels. Assuming putty is configured to log into bastion.hzg.de abd
forwards the local port 8000 to 141.4.0.159's port 9000, then the
configuration file should have no section Ssh and the host and port
entries in the section Network should read host~=~localhost, and
port~=~8000.

\section{Matool, the server}
The server is intended to be run on the dockserver. It probably makes
most sense to have it run by the user account localuser. 

When run, it will listen to port 9000 incomming
connections. Currently, there is no configuration file or command line
options to change this. There should be only one instance running of
the server. If run from the command line, it will first check whether
there are files already present in the to-glider directories and ask
whether they should be added to the queue. Basically that means that
the user tells the server whether or not the server should keep track
of the files in the to-glider directories.

The server is typically run from the command line by running the
command  \mts{}. A log file ma\_edit\_server.log is maintained in the
home directory. As the current version of the server is still in alpha
state, it is best to run the program in the foreground so that any
error messages generated can be used for debugging. If, accidentally,
a seconds instance of the server is run, it will abort complaining the
the address (port) is already in use. If no server is running, the
client will complain that there is no server available.

\end{document}
